\chapter*{General Conclusion}
\label{chap: General Conclusion}
\addcontentsline{toc}{chapter}{General Conclusion}


This project successfully developed and validated a Federated Learning (FL) system for distributed water quality monitoring, aiming to deliver a privacy-preserving, scalable, and real-time assessment solution. A multi-layer architecture was implemented, featuring simulated edge data collection, Raspberry Pi devices for local data preprocessing and model training, and a cloud backend for FL orchestration and system management. The collaboratively trained RNN model achieved an R² of 0.85 and an MAE of 3.5 for Water Quality Index prediction, with raw data always remaining local, and end-to-end functionality, including secure communications and web interfaces, was confirmed. However, the system's evaluation primarily used simulated data, and further validation is needed for real-world performance, large-scale scalability, handling data/model heterogeneity, and optimizing for edge device computational constraints.

Future work will prioritize deploying and rigorously testing the system in real-world conditions with physical sensors and a larger number of distributed clients to fully assess its scalability and robustness. Algorithmic enhancements will focus on incorporating more sophisticated preprocessing techniques and advanced FL strategies to better handle data and model heterogeneity. Further development will also include enriching the user interface capabilities and exploring the system's applicability to other environmental monitoring challenges, building upon the privacy-centric, decentralized foundation established by this project.